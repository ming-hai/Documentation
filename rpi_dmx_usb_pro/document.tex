\documentclass[titlepage,a4paper]{article}

\usepackage[a4paper, left=1.5cm, right=1.5cm]{geometry}
\usepackage{fancyhdr}
\usepackage{graphicx}
\graphicspath{ {./images/} }
\usepackage{hyperref}
\usepackage{listings}
\lstset{language=C}
\usepackage[singlelinecheck=false]{caption}
\usepackage{placeins}
\usepackage{parskip}

\title{Raspberry Pi DMX USB Pro}
\author{Arjan van Vught}
\date{\today}

\pagestyle{fancy}
\fancyhf{}
\fancyhead[L]{Raspberry Pi DMX USB Pro}
\fancyfoot[R]{Page \thepage}

\pagenumbering{roman}

\begin{document}

\maketitle

\tableofcontents
\listoffigures
\listoftables

\pagenumbering{arabic}

\break\section{Introduction}

\begin{tabular}{l p{12cm} }
\includegraphics[scale=0.5]{Rdm-logo-small.png} &
This article describes the Open source Raspberry Pi RDM Controller with USB,
DMX512 isolated board and compatible with software that supports Enttec USB
Pro.
\end{tabular}

\subsection{Main features}

\begin{itemize}

\item Raspberry Pi baremetal programming controls DMX512 / RDM timing
\item Open source \url{https://github.com/vanvught/rpidmx512/tree/master/rpi\_dmx\_usb\_pro}
\item Supported operating systems: Windows, MAC OS, Linux
\item Using the FT245RL USB to parallel interface for fast communication with the host 
\item DMX512 Transmitter or Receiver
\item No external power supply required
\item ANSI E1.11-2008 compliant (DMX512-A)
\item E1.20-2010 compliant (RDM)
\item Fully compatible with software suitable for ENTTEC USB Pro
\item Configuration is stored on the sdcard (file "params.txt")
\item Set your own manufacturer name and id (file "rdm\_device.txt")
\item Compatible with OpenLighting rdmpro\_sniffer (widget\_mode=3)
\item DMX receive rate throttling option to ensure receiving software isn't overloaded with too many frames per second 
\item Real-time statistics and process information is available on external monitor
\item UUID (or S/N) is based on MAC-address (works across all Raspberry Pi models)
\end{itemize}

\break\section{Configuration}

The \textbf{Widget configuration} is stored on the sdcard in the file
"params.txt"

\FloatBarrier
\begin{table}[h]
\begin{tabular}{l p{12cm}}
Widget Parameters & Description \\ \hline
dmxusbpro\_break\_time & DMX output break time in 10.67 microsecond units. Valid
range is 9 to 127. (default = 9) \\
dmxusbpro\_mab\_time & DMX output Mark After Break time in 10.67 microsecond
units. Valid range is 1 to 127. (default = 1) \\
dmxusbpro\_refresh\_rate & DMX output rate in packets per second. Valid range is
1 to 40, or 0 for fastest rate possible. (default = 40) \\
dmx\_send\_to\_host\_throttle & DMX receive rate throttling option in updates
per second. 0 is as fast as possible. (default = 0) \\
widget\_mode & The widget mode. See \ref{table:mode} for the supported values
\end{tabular}
\caption{paramt.txt}
\end{table}
\FloatBarrier

\FloatBarrier
\begin{table}[h]
\begin{tabular}{l p{12cm}}
Value & Description \\ \hline
0 & Mode 1 + 2 \\
1 & Normal DMX firmware. Supports all messages except Send RDM (label=7), Send
RDM Discovery Request(label=11) and receive RDM \\
2 & RDM firmware. This enables the Widget to act as an RDM Controller.
(default)\\
3 & RDM Sniffer firmware. This is for use with the Openlighting RDM packet
monitoring application.
\end{tabular}
\caption{Widget mode}
\label{table:mode}
\end{table}
\FloatBarrier
Example :
\begin{verbatim}
  #dmxusbpro_break_time=
  #dmxusbpro_mab_time=
  #dmxusbpro_refresh_rate=0
  widget_mode=0
  #dmx_send_to_host_throttle=20
\end{verbatim}

The \textbf{RDM device configuration} is stored on the sdcard in the file
"rdm\_device.txt"

\FloatBarrier
\begin{table}[h]
\begin{tabular}{l l}
Name & Description \\ \hline
manufacturer\_name & Manufacturer Name (maximum length is 32 characters) Widget Label = 77 \\ 
manufacturer\_id & Manufacturer ID : \url{http://tsp.plasa.org/tsp/working\_groups/CP/mfctrIDs.php} \\ 
device\_label & Descriptive label for the device (maximum length is 32 characters). Widget Label = 78 \\
\end{tabular}
\caption{rdm\_device.txt}
\end{table}
\FloatBarrier

Example:
\begin{verbatim}
  manufacturer_name=Open Lighting
  #manufacturer_id=7A70
  #device_label=
\end{verbatim}

\break\section{Firmware}

\break\section{Hardware}

\subsection{GPIO}

\subsection{USB}

The FT245RL USB to parallel interface is used for fast communication with the
host. Table \ref{table:ft245rl} refers to the connection details.

The 8 bits data for reading from the USB is available from bcm2835 register
GPLEV0. The data is read as follows:

\begin{lstlisting}
uint32_t in_gpio = (BCM2835_GPIO->GPLEV0 & 0b111110011100) >> 2;
uint8_t data = (uint8_t) ((in_gpio >> 2) & 0xF8) | (uint8_t) (in_gpio & 0x0F);
\end{lstlisting}

For writing the 8 bits data we need to mask the 0's and the 1's as follows:

\begin{lstlisting}
uint32_t out_gpio = ((data & ~0b00000111) << 4) | ((data & 0b00000111) << 2);
BCM2835_GPIO->GPSET0 = out_gpio;
BCM2835_GPIO->GPCLR0 = out_gpio ^ 0b111110011100;
\end{lstlisting}

Hence it is save to write 32-bits to bcm2835 register GPSET0. Quote from the
`BCM2835-ARM-Peripherals.pdf` : `The SET{n} field defines the respective GPIO
pin to set, writing a �0� to the field has no effect. If the GPIO pin is being
used as in input (by default) then the value in the SET{n} field is ignored.'
The same applies for writing to the bcm2835 register GPCLR0. Quote from the
`BCM2835-ARM-Peripherals.pdf` : `The CLR{n} field defines the respective GPIO
pin to clear, writing a �0� to the field has no effect.'

\FloatBarrier
\begin{table}[h]
\begin{tabular}{l l p{0.4cm} l}
RPi & SoC &  & USB \\
P1 & Broadcom & & FT245RL \\
\hline
3 & GPIO02 & $\leftrightarrow$ & D0 \\
5 & GPIO03 & $\leftrightarrow$ & D1 \\
7 & GPIO04 & $\leftrightarrow$ & D2 \\
26 & GPIO07 & $\leftrightarrow$ & D3 \\
24 & GPIO08 & $\leftrightarrow$ & D4 \\
21 & GPIO09 & $\leftrightarrow$ & D5 \\
19 & GPIO10 & $\leftrightarrow$ & D6 \\
23 & GPIO11 & $\leftrightarrow$ & D7 \\ \hline
15 & GPIO22	& $\rightarrow$ & WR \\
16 & GPIO23	& $\rightarrow$ & $\overline{\mbox{RD}}$ \\ \hline 
18 & GPIO24	& $\leftarrow$ & $\overline{\mbox{TXE}}$ \\
22 & GPIO25	& $\leftarrow$ & $\overline{\mbox{RXF}}$ \\
\hline
\end{tabular}
\caption{My table}
\label{table:ft245rl}
\end{table}
\FloatBarrier

\end{document}
