\documentclass[draft,titlepage,a4paper]{article}

\usepackage{fancyhdr}
\usepackage{hyperref}

\title{Raspberry Pi DMX USB Pro}
\author{Arjan van Vught}
\date{\today}

\pagestyle{fancy}
\fancyhf{}
\fancyhead[L]{Raspberry Pi DMX USB Pro}
\fancyfoot[R]{Page \thepage}

\pagenumbering{roman}

\begin{document}

\maketitle

\tableofcontents

\break

\pagenumbering{arabic}

\section{Introduction}

Open source USB DMX512 RDM and compatibility with software that supports Enttec USB Pro.

\subsection{Main features}

\begin{itemize}

\item Raspberry Pi baremetal programming controls DMX512 timing
\item Open source \url{https://github.com/vanvught/rpidmx512/tree/master/rpi\_dmx\_usb\_pro}
\item Supported operating systems: Windows, MAC OS, Linux
\item Using the FT245RL USB to parallel interface for fast communication with the host 
\item DMX512 Transmitter or Receiver
\item No external power supply required
\item ANSI E1.11-2008 compliant (DMX512-A)
\item E1.20-2010 compliant (RDM)
\item Fully compatible with software suitable for ENTTEC USB Pro
\item Configuration is stored on the sdcard (file "params.txt")
\item Set your own manufacturer name and id (file "rdm\_device.txt")
\item Compatible with OpenLighting rdmpro\_sniffer (widget\_mode=3)
\item DMX receive rate throttling option to ensure receiving software isn't overloaded with too many frames per second 
\item Real-time statistics and process information is available on external monitor
\item UUID (or S/N) is based on MAC-address (works across all Raspberry Pi models)
\end{itemize}

\section{Title new}

More text

\end{document}
